\documentclass[11pt]{article}
\usepackage{mgates-letter}
\definecolor{dark_blue} {rgb}{0., 0., 0.65}
\usepackage{makecell}
\usepackage[
backend=biber,
style=numeric,
citestyle=numeric,
maxcitenames=10,
maxnames=10,
%entrykey=false,
%annotation=false,
url=false
]{biblatex}

\usepackage{textcomp}
\usepackage{mathrsfs}  % mathscr font
\usepackage{boxedminipage}
\usepackage{rotating}
\usepackage{csquotes}
%\usepackage{natbib}
\usepackage[colorlinks, filecolor=dark_blue, urlcolor=dark_blue, linkcolor=black, citecolor=black]{hyperref}

\defbibcheck{mine}{\iffieldequalstr{annotation}{mine}{}{\skipentry}}
\defbibcheck{other}{\iffieldequalstr{annotation}{other}{}{\skipentry}}

\addbibresource{biblio.bib}

\begin{document}
\sloppy
\begin{center}
	{{
		\Large{
			\textsc{PhD Programme in Computer Science and Engineering \\ 
			\vspace{4mm}
			Cycle XXXVI}
			}
	}} 
	\rule[0.1cm]{\textwidth}{0.1mm}
	\rule[0.4cm]{\textwidth}{0.6mm}
\end{center}

\begin{center}
	{\LARGE{A Language-based Software Engineering Approach for Cyber-Physical Swarms}} \\
	\vspace{4mm}
	{\large{PhD Year III -- Report}} 
	\vspace{4mm}
\end{center}
\vspace{8mm}
\par
\noindent
\begin{minipage}[t]{0.47\textwidth}

{\large{Commission: \\\bf
Prof. Mirko Viroli \\
Prof. Andrea Omicini \\
Prof. Matteo Ferrara} 
}
\end{minipage}
\hfill
\begin{minipage}[t]{0.47\textwidth}
	\raggedleft
	{
		\large{PhD Student: \\\bf Gianluca Aguzzi}
	}
\end{minipage}
\vspace{10mm}

{
	\raggedright
	\rule[0.1cm]{\textwidth}{0.6mm}
	\rule[0.5cm]{\textwidth}{0.1mm}
}

\newcommand{\rev}[1]{{
	%\color{red}
	#1
	}}
\section{Introduction}
Over recent years, various research domains including pervasive~\cite{pervasive}, ubiquitous~\cite{weiser1999computer}, collective~\cite{abowd2016beyond}, and everyware computing~\cite{greenfield2010everyware} have been promoting a vision 
 that encompasses a multitude of computational devices like smartphones, 
 PCs, and embedded systems, working in unison to construct collective applications. 
Within this backdrop, my research refers to these expansive multi-agent systems as \textit{Cyber-Physical Swarms} (CPSWs).

The terminology draws inspiration from nature, 
 suggesting that this extensive network of computational devices can be viewed, 
 from a macroscopic standpoint, as a ``swarm''. 
% 
These swarms consist of basic units that engage in complex interactions to accomplish sophisticated collective tasks. 
 Additionally, these units are \textit{cyber-physical}, 
 meaning they possess a physical manifestation that allows them to interact with and modify the real world.

This conceptual framework is applicable to various types of systems, 
 including \textit{swarm robotics}, \textit{human crowds}, or more broadly, IoT device ``swarms''.

Creating applications for CPSWs presents numerous challenges, such as:
\begin{enumerate}
\item the inherently volatile nature of the environments these entities inhabit,
\item the potential for prolonged instability in collective behaviour, and
\item the absence of a centralized governing body for agent coordination.
\end{enumerate}

Current research efforts are focused on devising resilient, 
 efficient, and effective self-adaptive collective behaviors, 
 akin to those observed in natural settings. 
% 
My primary objective during my PhD is to develop a systematic approach—comprising models, 
 techniques, and algorithms—for generating and implementing self-organizing behaviours with predictable outcomes in CPSWs.

Specifically, my work adopts a language-centric methodology, constructing models and tools around a chosen programming language. I
 predominantly utilize aggregate computing (AC)~\cite{beal2015aggregate}, 
 a groundbreaking global-to-local programming paradigm. 
%
I opted for this language because its abstractions simplify the development of collective applications in the context of CPSWs, treating collective behaviour as evolving \textit{computational fields} that can be manipulated functionally and declaratively.

Although AC has already found applications in various contexts, 
 there remains a need to both venture into new research avenues and enhance existing solutions. 
 This is crucial for bridging the gap between theory and practice, enabling the use of AC in contemporary distributed applications. During my third year, I have continued to explore this research landscape, 
 investigating new API for expressing collective behaviours and going deeper into the integration of Machine Learning techniques in the AC stack.
\section{Research}

This year, 
 I was mainly focused on integrating Machine Learning techniques into the Aggregate Computing stack.
%
This integration has several benefits, 
 including improving the adaptivity of aggregate systems, 
 and separating functional specifications (i.e., aggregate code) from non-functional ones (e.g., consumption, convergence times, ...).
%
In parallel to these works related to the combination of AC and RL, 
 I have also explored the development of algorithms/patterns/methodologies 
 that can be applied in the context of CPWs.
In the following, I will describe each of the main papers that I carried on
 and I briefly summarize the main activities of my period abroad.
\subsection{Activities}
\begin{itemize}
	\item in \textit{\citefield{roadmap}{title}}~\cite{roadmap} I have drafted 
 a roadmap highlighting the possible directions to follow to combine Aggregate Computing with Machine Learning.
%
In the article, I present the main directions of this integration (Figure 1), which are:
\begin{itemize}
	\item learning aggregate programs/algorithms: designing efficient and versatile AC algorithms can be complex, 
	therefore, it is interesting to explore whether algorithms can be learnt or synthesised given high-level functional goals.
	
	\item Learning execution strategy: for a given AC program or algorithm, 
	 multiple execution strategies can be applied, 
	 affecting aspects like the scheduling
	 of computation rounds, 
	 the scheduling of communications, 
	 the retention of messages from neighbours. 
	%
	Novel approaches~\cite{zambonelli2021time} 
	 adapt the execution choices at runtime 
	 depending on factors which may include the speed of environmental change, 
	 the energy level of a device, incentives in volunteering settings, 
	 or the desired Quality of Service.
	Since it is hard to design static
	 or dynamic execution strategies able 
	 to adequately take into account all the factors and goals, 
	 it could be possible to let a system (and its components) 
	 learn how to efficiently execute algorithms according 
	 to a set of given high-level objectives.
	
	\item Learning Aggregate Computing deployments: a logical AC system consists of a logical network of
	 logical devices operating following a specified execution protocol. 
	As shown in recent work on pulverised architectures~\cite{Casadei2020}, 
	 it turns out that different application partitioning schemas and implementations of the digital thread associated 
	 with the aggregate system are possible, 
	 as well as different deployments of application components onto the available Information-Communication Technology (ICT) infrastructure. 
	In this context, ML could be injected to have 
	 the system learns itself what is a (locally) optimal deployment 
	 for an aggregate application.
\end{itemize}
\item Concerning the integration of learning in aggregate programming, the work 
 entitled \textit{\citefield{aguzzi2022towards}{title}}~\cite{aguzzi2022towards} 
 discusses the use of RL as a way to learn pieces of code 
 that may be sensitive to environmental conditions. 
In this work, I was inspired by program sketching 
 and I use a version of Q-learning~\cite{watkins1992q} for multi-agent systems (Hysteric Q-Learning~\cite{matignon2007hysteretic})
 to handle the complex interaction between the nodes. 
 
\item In \textit{\citefield{rl-middleware}{title}}~\cite{rl-middleware}, 
 I used RL to learn scheduling policies 
 in order to optimize consumption 
 while maintaining good responsiveness w.r.t. the collective behaviour.  
This work can be seen as an extension of \textit{\citefield{zambonelli2021time}{title}}~\cite{zambonelli2021time} 
 in which AC itself is used to define scheduling policies. 
With our approach, it is possible 
 to learn online the policy that can 
 maximize energy savings while maintaining low consumption.

%
\item In \textit{\citefield{swarm-clustering}{title}}~\cite{swarm-clustering}
 I explored a new clustering technique to track space-time evolving phenomena. 
% 
These clusters can then be used at the application level to perform 
 distributed decision-making and/or create related cluster-wise synthesise information.
Our approach, unlike standard approaches found in literature, 
 works in a distributed and iterative manner by exploiting 
 aggregate processes -- abstraction introduced in Aggregate Computing 
 that allows concurrent collective computations.

\item \textit{\citefield{domains}{title}}~\cite{domains} considers a design abstraction used to
 devise complex decentralized computation in complex collective ecosystems.
%
The main abstraction, called \textit{dynamic decentralization domains}, is used to
 create a region in space that opportunistically will be formed to support distributed 
 sensing and acting.
%
In the paper, I also show the implementation of that abstraction using Scala.

\item For what concern the engineering methodologies, in \textit{\citefield{engineering}{title}}~\cite{engineering}
 I examine the issues of engineering aggregate systems, outlining a research roadmap that exemplifies
 how the automation and methodological approach could improve the current
 trial-and-error approach.
%
\item in \textit{\citefield{scafi-paper}{title}}~\cite{scafi-paper}, I describe ScaFi, 
 a toolkit used for Aggregate Computing written in Scala.
ScaFi is not new, in fact, I am one of the main contributors to the software and I am the lead designer of
 ScaFi Web~\cite{aguzzi2021scafi} --- A playground used to try Aggregate Computing online.
We made this article to make ScaFi more accessible to a broad scientific community, 
 since, currently, it is still not vastly applied in the community of collective systems.
\end{itemize}
\subsection{Period Abroad}
In August, I started my period abroad 
 at Aarhus University under the supervision of Lukas Esterle.
%
During this period, I was mainly interested in applying Graph Neural Networks (GNNs)~\cite{scarselli2008graph}
 -- a neural network that works on graphs -- 
 with Aggregate Computing programs.
%
In particular, I used GNN to forecast 
 the state of natural phenomena tracked by agents of an aggregate system, and then, 
 using Aggregate Programming, I would like to coordinate the agent in a way 
 that they could follow the phenomena of interest.
%
In doing this, I applied a Centralised Training and Decentralised Execution (CTDE)~\cite{foerster2018deep} approach. 
%
In this way, the local features are learnt using global data, 
 and then the neural architecture can be broken down to each node of the system.
%
This could be done since GNN can be also seen as a local application of matrix multiplication.
%
This integration has a broader impact on the research of Aggregate Computing combined with Machine Learning. 
%
Indeed, this could be used in the work already 
 explored to extract a representation of the local state agent 
 --- that, in fact, I found particularly hard using hand-craft solutions. 

\section{Talks}
\begin{enumerate}
	\item Towards Reinforcement Learning-based Aggregate Computing @ COORDINATION 2022, 13 - 17 June 2022, Lucca
	\item Machine Learning for Aggregate Computing: a Research Roadmap @ DISCOLI 2022, 10 - 13 July 2022, Bologna
	\item Addressing Collective Computations Efficiency: Towards a Platform-level Reinforcement Learning Approach @ ACSOS 2022, 19 - 23 September 2022, Online
\end{enumerate}
\section{International Conference Activities}
\begin{enumerate}
	\item Artifact Evaluation Committee @ COORDINATION 2022
	\item Student Volunteer @ ICDCS 2022
	\item Student Volunteer @ ACSOS 2022
	\item Sub-reviewer @ COORDINATION 2022
	\item Sub-reviewer @ AAMAS 2022
	\item Sub-reviewer @ DISCOLI 2022
	\item Sub-reviewer @ ASE NIER 2022
\end{enumerate}
\section{Teaching}
\begin{enumerate}
	\item Tutor @ Padigmi di Progettazione e Sviluppo (PPS), Ingegneria e Scienze Informatiche, 30 ore
	\item Tutor @ Programmazione Concorrente e Distribuita (PCD), Ingegneria e Scienze Informatiche, 30 ore
	\item Seminar entitled ``Scala: a Cross-Platform Language'' @ PPS
	\item Seminar entitled ``Scala(e) to the large. Concurrent programming in Scala and relevant Frameworks'' @ PPS
	\item Seminar entitled ``Akka: An introduction'' @ PCD
	\item Seminar entitled ``Akka for Distributed Systems'' @ PCD
	\item Lesson entitled "Reinforcement Learning, an introduction" @ ASAI Summer School 2023
\end{enumerate}
\section{Courses and School}

\begin{table}[H]
	\resizebox{\textwidth}{!}{%
	\begin{tabular}{|c|c|c|c|c|c|}
	\hline
		Professor & Course & Kind & Credits & Period & Exam \\ \hline
		Enrico Gallinucci & From Big Data to Data Platform & PhD Course & \makecell{10 hours \\ 2 proposed credits} & \makecell{January \\ 2022} & Not done \\ \hline
		Roberto Casadei & \makecell{Engeneering Collective Intelligence} & PhD Course & \makecell{10 hours \\ 2 proposed credits} & \makecell{December \\ 2020} & Done \\ \hline
		Marco Gori & \makecell{Towards Developmental Machine Learning} & \makecell{BISS 2022 \\ Spring School} & \makecell{10 hours \\ 4 credits } & \makecell{April \\ 2022} & Done \\ \hline
		Massimo Villari & \makecell{From Cloud to Serverless through microelements} & \makecell{BISS 2022 \\ Spring School} & \makecell{10 hours \\ 4 credits } & \makecell{April \\ 2022} & Done \\ \hline
		Armir Bujari & \makecell{Industry 4.0 and the Industrial Internet of Things: \\ Challenges and Enabling Technologies} & \makecell{PhD Course} & \makecell{15 hours \\ 3 credits } & \makecell{February \\ 2022} & Done \\ \hline
		Aristides Gionis & \makecell{Opinions and conflict in social networks: \\ models, computational problems, and algorithms} & \makecell{BISS 2022 \\ Spring School} & \makecell{X} & \makecell{February \\ 2022} & Not Done \\ \hline
		X & \makecell{MOOC \\ Introduction to Complexity \\ \href{https://www.complexityexplorer.org/courses/119-introduction-to-complexity-2021}{\textbf{MOOC}}} & \makecell{MOOC} & \makecell{15 hours \\ 3 proposed credits} & \makecell{September \\ 2022} & \makecell{Done} \\ \hline
	
	\end{tabular}
	}
\end{table}

\section{Papers}
\subsection{Published}
\begin{itemize}
	\item \fullcite{swarm-clustering}
	\\{\footnotesize\textbf{Abstract:} \citefield{swarm-clustering}{abstract}}
	\item \fullcite{aguzzi2022towards}
	\\{\footnotesize\textbf{Abstract:} \citefield{aguzzi2022towards}{abstract}}
\end{itemize}
\subsection{Accepted}
\begin{itemize}
	\item \fullcite{scafi-paper}
	\\{\footnotesize\textbf{Abstract:} \citefield{scafi-paper}{abstract}}
	\item \fullcite{roadmap}
	\\{\footnotesize\textbf{Abstract:} \citefield{roadmap}{abstract}}
	\item \fullcite{engineering}
	\\{\footnotesize\textbf{Abstract:} \citefield{engineering}{abstract}}
	\item \fullcite{rl-middleware}
	\\{\footnotesize\textbf{Abstract:} \citefield{rl-middleware}{abstract}}
\end{itemize}
\subsection{In Peer Review}
\begin{itemize}
	\item \fullcite{domains}
	\\{\footnotesize\textbf{Abstract:} \citefield{domains}{abstract}}
\end{itemize}
\section{References}
\printbibliography[heading=none]

\end{document}
